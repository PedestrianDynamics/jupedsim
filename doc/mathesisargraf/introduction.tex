\section{Pedestrian Dynamics: Introduction}
%<< big picture: micor-/macroscopic models, cell automata/ODE-based, take a closer look in next chapter >>

Pedestrian dynamics defines a field of research trying to understand the kinematic and mechanic of pedestrian crowd movement. Understanding, how crowds react in different geometries under various circumstances, enables a safer design of our environment, to best fit the needs of civil and security engineering. Results are applied to safely conduct large events, to create architecture, through which large crowds can safely be moved and to optimize evacuation time in case of an emergency. At the annual hajj\footnote{Islamic pilgramage to Mecca, Saudi-Arabia.} in Mecca in 2015, a tragic panic started, when two crowds from opposite directions met on a street, 12 meter wide. Saudi authorities state that more than 700 pilgrims died in a stampede and at least 860 more had been injured.\footnote{BBC News, September 25th}

Pedestrian dynamics provides approaches to plan large events by calculating estimates for capacities of given geometries, researching crowd behavior and applying research results in new designs of civil engineering.
To simulate pedestrian crowds, many models exist with different characteristics. Predtetschenskii and Milinkskii \citep{Predtechenskii1971} are pioneers in pedestrian dynamics, conducting experiments as early as 1969. Few years later, Hirai and Tarui \citep{Hirai1975} implemented the first force-based model to simulate crowd behavior. Since then, new models have been described throughout the decades. To maintain orientation, these models can be grouped into classes in the following manner (see figure \ref{classes}):

\begin{figure}[h!]                                                                                      
  \begin{center}
  \tikzstyle{block} = [draw, rectangle, rounded
  corners, minimum height=2em,fill=blue!10,text width=8em, text centered]                                                                                                                                                                                                                    

  \begin{tikzpicture}[auto, node distance=4.5cm,>=latex', ultra thick]
  \node [block, name=macro] {Macroscopic};
  \node [block, name=micro, right of=macro] {Microscopic};
  \node [block, name=meso, right of=micro] {Mesoscopic};

    %\node [block, name=dis, below= 0.8cm of macro] {Discrete};
    %\node [block, name=con, right of=dis] {Continuous};
    %\node [block, name=mix, right of=con] {Mixed};
   
    \node [block, name=rule, below= 0.8cm of macro] {Rule-based};
    \node [block, name=ode, right of=rule] {ODE-based};
    \node [block, name=hybr, right of=ode] {Hybrid};   
    
    \node [block, name=force, below = 0.8cm of rule] {Force-based};
    \node [block, name=vel, right of=force] {Velocity-based};
    \node [block, name=other, right of = vel] {other};       

    \draw [->] (micro) -- (rule);
    \draw [->] (micro) -- (ode);
    \draw [->] (micro) -- (hybr);
   
    %\draw [->] (con) -- (rule);
    %\draw [->] (con) -- (ode);
    %\draw [->] (con) -- (hybr);                                                  

	\draw [->] (ode) -- (force);
	\draw [->] (ode) -- (vel);
	\draw [->] (ode) -- (other);
                                                    
\end{tikzpicture}                                                                                                         
\end{center}
\caption{A possible hierarchical classification of models in pedestrian dynamics in \citep{Chraibi2012}}
\label{classes}
\end{figure}

\emph{Macroscopic} models tackle crowd behavior without the need to characterize individuals, which make up the crowd. The action of a single agent is neglected and it is assumed, that aggregated values are sufficient to describe the crowd behavior. Metrics, e.g. density or flow, are used to describe the dynamic within the system. Thus a crowd is seen as a continuous fluid, which can be modeled by these aggregated observables only. No inter-particle relations are explicitly considered. Given a model, which describes the change of the density throughout a geometry, it can be mathematically captured by an PDE.   Larger roadmap- and city-traffic-simulation are fields, where macroscopic models are widely spread and can supply travel times and point out bottlenecks \citep{Emme}. There are limitations to this class of models. They are fast but lack the ability to simulate heterogeneous groups. Nor can they model individual decisions. In a panic situation, it has been observed, that pedestrians follow the crowd, even if other exits are available. A flow-based model would have the pedestrians use all available exits \citep{Marno}.

\emph{Microscopic} models consist of mathematical formulations describing the state and the interactions of every agent. Each agent has a position in the domain and interacts with its environment. It is assumed that the dynamics in any crowd is the result of individual actions. Within the model, these individual actions obviously must be different from the attempt to model the complete, complex system of a person's psychology, which defines its motivation of movement inside a crowd. It is desirable to have few and simple equations to model the agent's motivation. In analogy to Newtonian dynamics, it can be modeled by driving and repelling forces \citep{Helbing2001}. They lead to second order ordinary differential equations.
A popular starting point origins in the modeling of the behavior of electrical charges in an electro-magnetic potential field (see figure \ref{forcesOnAgent}). Charges of the same sign act on each other with a repelling force. This effect is used in the modeling of the natural collision-avoidance of a person to other persons, walls and obstacles in pedestrian dynamics. Superposing a driving force, that acts on the agent, steering it towards its destination, the resulting \emph{force based} model can be described by an ODE.\footnote{This simplified describtion shall suffice for this introduction. For further reference, please see corresponding literature. A comprehensive insight in SFMs is given in \citep{Chraibi2012}}
\begin{figure}[h!]
\input{forcesOnAgent.pdf_t}
%\includegraphics[width=1.0\linewidth]{pics/forces_on_agent.pdf}
\caption{Forces acting on agent A from: wall, obstacle and agent B}
\label{forcesOnAgent}
\end{figure}
\newline Force-based and velocity-based models are subject of this thesis, so we turn to ODE-based models, which is the super-class to both.

\subsection{ODE based, microscopic models}

We focus on ODE-based microscopic models, which are successful in producing system phenomena like congestions in front of bottlenecks and showing good accordance with experimentally determined fundamental diagrams
\footnote{Fundamental diagrams plot the (pedestrian) flow over the (pedestrian) density.}
[Dietrich, 2014]. Agents seem to sway left and right and do not steer as wanted. Various authors try to find new models with enhanced characteristics in terms of directing agents \citep{Moussaid2011}\citep{Chraibi2011}.

The difficult calibration of a model is another important issue. This type of models find their limit, if one is to find one constant set of parameters for various situations. Best results are achieved with a special calibrated set of parameters for each situation. The change of parameter sets would be problematic, if we want to make extensive use of parallel solvers. In terms of ergonomics, a constant set would be more user-friendly. 

\emph{Velocity based} models, which often lead to first order ODEs, make up an important sub-group. These models change the agent's velocity directly and thus show much better trajectories in terms of oscillation. The TEST\footnote{substitute name of model}-model of this thesis, is partly derived from the \emph{GNM} \citep{Dietrich2014}, a velocity-based model. Dietrich motivates the creation of his model. He intends to overcome short-comings of both groups, oscillation (SFM) and the difficult mathematical treatment of Optimal Step Models, yet have the positive characteristics remain.
A major step towards this goal is the use of a navigation field, the solution to the Eikonal equation. This approach, introduced by Hartmann \citep{Hartmann2010}, provides routing and navigation information. In the GNM, Dietrich divides the navigation into two components. A static and a dynamic navigation vector are described. The static navigation field comprehends the geometry, the dynamic navigation field integrates pedestrians and mobile obstacles. It is clear, that the dynamic navigation field must be computed for every time-step throughout the simulation.

Besides oscillation and calibration, there is a third issue, on which we will focus in the next chapter: Overlapping. It describes a situation, where an agent's position is invalid either because the agent overlapps with another agent or because his simulated presence overlapps with a wall or even an obstacle. Once an agent is fully clipped through a wall-surface, faulty trajectories are most certain.

These issues highlight the need to develop pedestrian models and search for yet another model, which might overcome some of the shortcomings and can produce as good results as existing models already provide.