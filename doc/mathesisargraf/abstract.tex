\section*{Abstract}
%In dieser Arbeit wird genommen: Model von Felix, Bodenfeld von Madrid, Code adaptiert von Leonardo Andrés Zepeda Núñez. 
%Es wird ein Testmodel beschrieben, welches den Einfluß des Madrid-Bodenfeldes untersucht.
%Historie: Dietrich's Model in Jupedsim; Bodenfeld warf Frage auf: Welche Diskretisierung (rectGrid, triangulated, how to deal with wall-surfaces), Probleme durch non-smooth Bodenfeld (?); -> Beschluss: Bodenfeld so gestalten, dass es gute Eigenschaften bei der Fußgängersimulation verspricht. 
%Dieses dann untersuchen und qualifizieren. Durch die Betreuung von M.C. floss die Erfahrung zahlreicher Modelle ein und es gelang bei der Modelfindung ein geeignetes Testmodel neu zu beschreiben.

In this thesis, the effect of an alternate floor-field is analyzed, by using it in a newly composed test-model for pedestrian dynamics. In simulations of pedestrian movement, the routing of agents\footnote{An agent is the representation of a pedestrian in the simulation.} is an integral part. Routing can be seen as the composition of two aspects: the global pathfinding through a geometry and the avoidance of static or dynamic obstacles (like walls or other agents) in a local situation.

Development of pedestrian simulation shows various models with different answers to the question of navigation. Many of which make use of manually added elements\footnote{like some sort of domain-decomposition, e.g. through helplines} to solve the global pathfinding, which enable the user to simulate crowd movement in that specific geometry. Other models use an algorithm, that will supply a navigation direction for any geometry. The \emph{Gradient Navigation Model}(GNM) described by Dietrich\citep{Dietrich2014} is one of the later. It uses the solution of the Eikonal equation (see chapter \ref{eikonalequation}), which describes a 2-D wave-propagation. The wave starts in the target region and propagates throughout the geometry. Agents are directed in the opposite direction of the gradient of aforementioned solution of the Eikonal equation. The Routing using the plain floor-field will yield non-smooth pathways. This could pose a problem for models, relying on a well-posed problem: In \cite{Dietrich2014}, Dietrich shows the existance and uniqueness of a solution to his problem-formulation by using the theorem of Picard-Lindel{\"o}f.
%\footnote{Picard-Lindelöf theorem: Consider the initial value problem \newline \begin{center}
%$y'(t) = f(t,y(t)), \quad y(t_0 ) = y_0, \quad t \in [t_0 - \epsilon, t_0 + \epsilon]$.\newline
%\end{center}  Suppose $f$ is uniformly Lipschitz continuous in y and continuous in $t$. Then, for some value $\epsilon > 0$, there exists a unique solution $y(t)$ to the initial value problem on the interval $[t_0 - \epsilon, t_0 + \epsilon]$.}
To apply this theorem, Lipschitz-continuous first derivatives of the input-functions must exist. This contradicts with non-smooth pathways in a plain floor-field. He solves that problem by the use of a mollifier, which basically takes a locally integrable function and returns a smooth approximation. Thus he creates a well-posed problem.

In this thesis, an enhanced floor-field is described, which addresses aforementioned issue (non-smoothness) as a welcome side-effect. A research-group at the Universidad Carlos III de Madrid \citep{Madrid} is working on safe navigation of robots. Since agents should not follow paths, which come close to any obstacle, a distance-field is created and used in the Fast-Marching algorithm, resulting in \emph{smooth} pathways, which favor a certain distance to walls. The researchers take that approach even further, by transforming any geometry into a skeleton (again using the distance-field) and thus having the domain in which the 2-D wave propagates reduced dramatically. Their intent is to re-calculate the floor-field in real-time using it for the reduced view-field of a robot's sensors.
Our interest in this sleight of hand is different. We take special interest in the behavior of agents close to obstacles. The enhanced floor-field itself yields pathways, which show a wall-repulsive character in the negative gradient close to walls. This component enables us to formulate a new model. It is implemented in JuPedSim\cite{jupedsim}, a simulation suit for pedestrian simulation, developed at the J{\"u}lich Supercomputing Centre, Forschungszentrum J{\"u}lich GmbH. It is verified and validated with respect to empirical data. The results seen in the simulations show remarkably good behavior. The model is easy to use, fast and shows an organic routing through complex geometries. The extent to which we alter the floor-field is subject to our analysis.
