% Class diagram
% Author: Remus Mihail Prunescu
\documentclass{minimal}
\usepackage[a4paper,margin=1cm,landscape]{geometry}
\usepackage{tikz}

%%%<
\usepackage{verbatim}
\usepackage[active,tightpage]{preview}
\PreviewEnvironment{tikzpicture}
\setlength\PreviewBorder{5pt}%
%%%>

\begin{comment}
:Title:  Class diagram

\end{comment}
\usetikzlibrary{positioning,shapes,shadows,arrows}

\begin{document}
\tikzstyle{abstract}=[rectangle, draw=black, rounded corners, fill=blue!40, drop shadow,
        text centered, anchor=north, text=white, text width=4.5cm]
\tikzstyle{comment}=[rectangle, draw=black, rounded corners, fill=green, drop shadow,
        text centered, anchor=north, text=white, text width=3cm]
\tikzstyle{myarrow}=[->, >=open triangle 90, ultra thick]
\tikzstyle{line}=[-, ultra thick]
        
\begin{center}
\begin{tikzpicture}[node distance=2cm]
    \node (Geometry) [abstract, rectangle split, rectangle split parts=2]
        {
            \textbf{Geometry}
        };
    \node (Rooms) [abstract, rectangle split, rectangle split parts=2, below=of Geometry]
        {
            \textbf{Rooms}
            % \nodepart{second}nil
        };
    \node (Subrooms) [abstract, rectangle split, rectangle split parts=2, below=of Rooms]
        {
            \textbf{Subrooms}
            % \nodepart{second}nil
        };
    \node (AuxNode02) [text width=0.5cm, below=of Subrooms] {};
    \node (Transitions) [abstract, rectangle split, rectangle split parts=2, right=of Geometry]
        {
            \textbf{Transitions}
            \nodepart{second}connecting rooms
        };
    \node (Crossings) [abstract, rectangle split, rectangle split parts=2, right=of Rooms]
        {
            \textbf{Crossings}
            \nodepart{second}connecting  subrooms
        };

    \node (Walls) [abstract, rectangle split, rectangle split parts=2, left=of AuxNode02]
        {
            \textbf{Walls}
            % \nodepart{second}nil
        };
        \node (Obstacles) [abstract, rectangle split, rectangle split parts=2, right=of AuxNode02]
        {
            \textbf{Obstacles}
            % \nodepart{second}nil
        };

    \draw[myarrow] (Rooms.north) -- (Geometry.south);
    \draw[myarrow] (Transitions.west) -- (Geometry.east);
    \draw[myarrow] (Subrooms.north) -- (Rooms.south);
    \draw[myarrow] (Crossings.west) -- (Rooms.east);
    \draw[myarrow] (Walls.north)  ++(0,0.8) -|  (Subrooms.south)  ;
    \draw[line] (Walls.north)  ++(0,0.8) -|  (Obstacles.north);
    \draw[line] (Obstacles.north)  ++(0,0.8) -|  (Walls.north);
        
\end{tikzpicture}
\end{center}
\end{document}
%%% Local Variables:
%%% mode: latex
%%% TeX-master: t
%%% End:
